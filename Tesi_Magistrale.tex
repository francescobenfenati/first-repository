\documentclass[12pt,a4paper,twoside,openright]{report}

%\documentclass[a4paper,11pt]{report}
\usepackage[T1]{fontenc}
\usepackage[latin1]{inputenc}
\usepackage[english]{babel}
\usepackage{comment}
\usepackage{quoting}
\usepackage{graphicx}
\usepackage{amsmath}
\usepackage{cases}
\usepackage{amsthm}
\usepackage{mathtools}
\usepackage{version}
\usepackage{amssymb}
\usepackage{ytableau}
\usepackage{booktabs}
\usepackage{tabularx}
\usepackage{array}
\usepackage{braket}
\usepackage[binding=1cm]{layaureo}
\usepackage[nottoc,notlot,notlof]{tocbibind}
\usepackage{subfig}
\usepackage{bm}
\usepackage{hyperref}

%\usepackage[colorlinks=true, linkcolor=blue]{hyperref}
\quotingsetup{font=small}

\newcommand{\numberset}{\mathbb}
\newcommand{\N}{\numberset{N}}
\newcommand{\R}{\numberset{R}}
\newcommand{\C}{\numberset{C}}
\newcommand{\bigequal}{\scalebox{1.9}{\ensuremath =}}
 
\pagestyle{headings}



\usepackage[english]{babel}
\usepackage{newlfont}
\usepackage{color}
\textwidth=450pt\oddsidemargin=0pt


\begin{document}
\begin{titlepage}
%
%
% UNA VOLTA FATTE LE DOVUTE MODIFICHE SOSTITUIRE "RED" CON "BLACK" NEI COMANDI \textcolor
%
%
\begin{center}
{{\Large{\textsc{Alma Mater Studiorum $\cdot$ Universit\`a di Bologna}}}} 
\rule[0.1cm]{15.8cm}{0.1mm}
\rule[0.5cm]{15.8cm}{0.6mm}
\\\vspace{3mm}

{\small{\bf Scuola di Scienze \\ 
Dipartimento di Fisica e Astronomia\\
Corso di Laurea in Fisica}}

\end{center}

\vspace{23mm}

\begin{center}\textcolor{black}{
%
% INSERIRE IL TITOLO DELLA TESI
%
{\LARGE{\bf Qualcosa su KM3NeT}}\\
}\end{center}

\vspace{50mm} \par \noindent

\begin{minipage}[t]{0.47\textwidth}
%
% INSERIRE IL NOME DEL RELATORE CON IL RELATIVO TITOLO DI DOTTORE O PROFESSORE
%
{\large{\bf Relatore: \vspace{2mm}\\\textcolor{black}{
Prof.ssa Annarita Margiotta}\\\\
%
% INSERIRE IL NOME DEL CORRELATORE CON IL RELATIVO TITOLO DI DOTTORE O PROFESSORE
%
% SE NON AVETE UN CORRELATORE CANCELLATE LE PROSSIME 3 RIGHE
%
\textcolor{black}{
\bf Correlatore: 
\vspace{2mm}\\\
Dr. Tommaso Chiarusi\\\\}}}

\end{minipage}
%
\hfill
%
\begin{minipage}[t]{0.47\textwidth}\raggedleft \textcolor{black}{
{\large{\bf Presentata da:
\vspace{2mm}\\
%
% INSERIRE IL NOME DEL CANDIDATO
%
Francesco Benfenati}}}
\end{minipage}

\vspace{40mm}

\begin{center}
%
% INSERIRE L'ANNO ACCADEMICO
%
Anno Accademico \textcolor{black}{ 2020/2021}
\end{center}

\end{titlepage}





%\title{Modello Statico a Quark degli Adroni Leggeri}
%\author{Francesco Benfenati}
%\date{}
%\maketitle
\begin{abstract}
The newborn Neutrino astronomy is one of the most promising branches of astroparticle physics. Its main goal is to measure the fluxes of astrophysical neutrinos and identifying their sources, thus allowing us to achieve a better understanding of the sources and the acceleration mechanisms of Cosmic Rays. As a matter of fact, due to their unique physical properties neutrinos constitute an optimal probe to observe high-energy astrophysical phenomena: they interact exclusively via the Weak Interaction and they are of neutral charge. These properties grant them the capabilities of propagating throughout the Universe for very large distances without being absorbed, in this sense differing a lot from photons, and of being not deflected by magnetic fields in their travel. For these reasons Neutrino astronomy is expected to become decisive for settling questions about Cosmic Ray sources which cannot be solved even by $\gamma$-ray astronomy.

From the first time that M. Markov proposed, in the '60s, the idea of Neutrino telescopes conceived as ``\emph{detectors deep in a lake or in the sea, for  determining the direction of the charged particles with the help of Cherenkov radiation}'' huge steps forwards have been made: the IceCube neutrino telescope, built in the Antarctica and completed in 2011 has already proven to be working obtaining the first measurement of an astrophysical high-energy neutrino flux and in 2017 the first ever evidence of a neutrino coming from an astrophysical source. Nowadays the second generation of neutrino telescopes like KM3NeT, IceCube-Gen2, GVD is under construction or development. All of these experiment share the objective of identifying with an unprecedented precision astrophysical sources of high-energy neutrinos, exploiting the Multi-messanger strategy, that is by correlating their location obtained by neutrino data with data coming from $\gamma$-ray or gravitational waves' measurement.  

In particular, the KM3NeT-ARCA detector of the KM3NeT experiment is under construction and it is going to be installed in the depths of the Italian Mediterranean Sea. In its final configuration it will reach a volume greater than $1 \,\textup{km}^3$ of sea water, instrumented with thousands of optical sensors, thus becoming the most sensitive high-energy neutrino telescope, with a substantial improvement with respect to the IceCube experiment's sensitivity. Moreover, being located in the Northern hemisphere, it will be able to observe the fraction of sky which includes the centre of our Galaxy, a region where most of neutrino sources are expected to live.
\end{abstract}

\tableofcontents
%\bibliographystyle{plain}
%\bibliography{biblio}


\chapter{Neutrino astronomy} \label{Chapter1}
One of the major topics in Astrophysics is the one related to Cosmic Rays (CRs). While to date we have learned a lot about the nature and the composition 
of these rays, we still have important gaps in our knowledge of their astrophysical sources and their acceleration mechanisms.
With this respect, significative progresses have been made in the last decades with the advent of the first generation of experiments in Space and on Earth able to measure the high-energy and ultra-high-energy range of the $\gamma$-ray spectrum. As we will see in this chapter, the production of these $\gamma$-rays is foreseen in the acceleration and propagation mechanisms of charged CRs, both of electrons and of protons or heavy nuclei (Z > 3). The importance of this neutral probes stains in the fact that its travel towards us follows a straight line, since it is not deflected by the Galactic or Intergalactic magnetic fields, allowing us to trace back directly its production point. The opening of this new window on the CRs study represented a breakthrough in our  comprehension of the subject, and from the results of $\gamma$-ray measurements we have discovered several classes of either Galactic or Extragalactic cosmic rays accelerators and sources. 

Nevertheless, there are still open questions to which we cannot answer by relying exclusively on the use of photons as probes.
For example, since both the astrophysical sources which produce and accelerate electrons and the ones which produce and accelerate protons and heavy nuclei are linked to $\gamma$-ray production, we need to detect another kind of probes in order to distinguish between them. These can be high-energy neutrinos, which can be produced exclusively in processes involving hadrons. Neutrinos share with photons the nature of neutral particles and the advantage of following a straight track between the production and observation points, but different to photons their interactions in the propagation medium are much less probable, thanks to their intrinsically low cross sections. 
Therefore, while high-energy photons, starting from an energy of several TeV, interact with the infrared background radiation and the Cosmic Microwave Background radiation in the Universe producing an electron-positron pair, and they are partially absorbed by interstellar dust, neutrinos' absorption from the material across their path in the Universe is much more limited and we can detect neutrinos coming even from very far distances, from extremely dense regions of galaxies or from the innermost structure of astrophysical objects. In other cases they carry complementary informations with respect to the ones brought by photons, having a crucial role in the so-called ``Multimessenger Astrophysics''.

On the other hand, their small interaction probability represents an issue for their detection in an experiment. This is the reason why large volume detectors with very large masses of target material are needed for detecting them. The necessary technological advancements needed for building such detectors have been made available only in the last decades, and neutrino telescopes represent one of the most recent and promising branches of Astroparticle physics.
In this chapter a brief and non-exaustive review of the experimental status of the high-energy astroparticle physics and its theoretical fundaments is shown, with particular focus on neutrino's properties and neutrino astronomy.

\section{Cosmic Rays}\label{CosmicRays}
On Earth, we are continuously bombarded by an isotropic flux of high-energy charged particles coming from the sky. This has been known since the beginning of the last century, when Victor Hess conducted several balloon flights at different heights measuring the rate of ionization with an electroscope and discovered that ionization increased with height, leading him to the conclusion that the ionization was originated by radiation coming from space, which will be called in the following years \emph{Cosmic Rays}. For this discovery he was awarded with the Nobel Prize in 1936. Since those first experiments, an enormous amount of effort, both theoretical and experimental, has been spent in the quest for the origin of this radiation and for the investigation of its nature, producing giant leaps forwards in our knowledge of Cosmic Rays, but still leaving us with several unanswered questions.

Nowadays we know that Cosmic Rays are an isotropic flux of high-energy and stable particles originating and accelerated in astrophysical environments. They are classified as \emph{primary} and \emph{secondary} CRs. Primary cosmic rays are protons, fully ionized atomic nuclei and electrons. Part of this radiation can interact either in or near the acceleration regions or during its propagation producing high-energy charged or neutral secondary particles. Among them, $\gamma$-rays and neutrinos are stable and can arrive on Earth together with primary particles. Antiparticles, mainly positrons and antiprotons, are produced and are present in the cosmic radiation as well. Instead, secondary CRs are all particles produced far from the acceleration regions, in interactions occurring during the propagation of primary CRs in space or in the Earth's atmosphere. While primary CRs can be detected by detectors and telescopes in space orbiting the Earth, experiments on Earth can study the flux of arriving CRs only in an indirect way by measuring the secondary particles in the showers produced by the interactions of primary CRs in the atmosphere, and they must have a completely different concept with respect to direct CRs detection experiments.

The chemical composition of primary CRs of energy smaller than about $10^{14}\,\textup{eV}$ has been measured with the abundance of nuclei detected by space-born experiments ~\cite{Boezio:2012rr} and most estimations agree on a composition dominated mainly by protons ($\sim 85\%$) and Helium nuclei ($\sim13\%$) followed by heavy nuclei ($\sim1\%$), electrons ($\sim 1\%$) and antiparticles ( less than 1\%). The chemical composition of CRs of higher energy is still largely unknown due to the difficulties in determining the atomic number of the primary cosmic ray from the measurement, made by indirect cosmic ray detection experiment, of the secondary particles produced in the shower they develop in the atmosphere. 

\subsection{Cherenkov radiation}

The \emph{Cherenkov radiation} is an effect arising when a charged particle propagates in a medium of refractive coefficient $n$ with a velocity $v = \beta c$ which is larger than the speed of light in that medium $c/n$, it emits photons collimated at a specific angle called the \emph{Cherenkov angle}. As a matter of fact, the passage of such a particle polarizes the atoms along its path in an asymmetrical arrangement, inducing a net dipole moment which causes emission of Cherenkov photons when those atoms return back to the initial configuration. In such cases, a coherent conical wavefront is emitted, at a well-defined angle:

\begin{equation}
\cos\theta_C = \frac{1}{\beta n}
\end{equation}
with respect to the particle's trajectory. This leads to a threshold velocity $\beta_{thr}$ for the incoming particle equal to
\begin{equation}
\beta_{thr} = \frac{1}{n}
\end{equation}
thus implying a threshold energy for the incoming particle of 
\begin{equation}
E_{thr} =\frac{mc^2}{\sqrt{1-1/n^2}}
\end{equation}
where $m$ is the mass of the particle. The corresponding kinetic energy threshold $T_{thr}$ is then just $T_{thr} = E_{thr} - mc^2$. This situation is schematized in Fig.\,\ref{Fig:Cherenkov}.


\begin{figure}[]
\centering
	\includegraphics[scale=0.6]{Cherenkov_angle.pdf}
\caption{Scheme of the Cherenkov radiation. The propagation direction of the charged particle is represented together with the Cherenkov angle $\theta_C$ and the wavefronts formed by all the spherical lightwaves produced during the passage of the particle.}
\label{Fig:Cherenkov}
\end{figure}


The number of photons \cite{Leo:techniques} emitted per unit wavelength $d\lambda$ per unit path $dx$ as a particle passes through the radiating medium is:

\begin{equation}
\frac{d^2N}{d\lambda dx} = \frac{2\pi z^2 a}{\lambda^2} \Bigl (1-\frac{1}{\beta^2 n^2(\lambda)} \Bigr )
\label{eq:cherenkov_angle}
\end{equation}
where $\alpha$ is the fine structure constant and $z$ the atomic number of the radiating medium.

Let us consider now sea water as medium. 
In the wavelength range from 300 nm to 600 nm, in which water is transparent, the number of emitted photons per unit path length, obtained by integrating \ref{eq:cherenkov_angle} is about 
\begin{equation}
\frac{dN}{dx} \simeq 340 \,\textup{photons/cm}
\end{equation}
In this wavelength region, the refractive index of water is $n\approx 1.35$, which for a highly relativistic particle with $\beta \simeq 1$ results in a Cherenkov angle $\theta_C \simeq 42�$. The kinetic energy threshold corresponds to the values of $T^e_{thr}\simeq 0.25 \,\textup{MeV}$ for electrons, $T^e_{thr}\simeq 53\,\textup{MeV}$ for muons, and $T^e_{thr}\simeq 450 \,\textup{MeV}$ for protons.

\subsubsection{Water properties}
Light propagation in a medium is affected by absorption and scattering of photons. These affect the reconstruction capabilities of the telescope, since absorption reduces the number of photons arriving on PMTs, whereas scattering changes the direction of Cherenkov photons, consequently delaying their arrival time on the PMTs: this results in a degraded measurement of the direction of the incoming particle.
In order to describe these effects, we introduce two parameters: the \emph{absorption length} $\lambda_{abs}$ and the \emph{scattering length} $\lambda_s$.
Each of these quantities represents the path after which a beam of light of initial intensity $I_0$ and wavelength $\lambda$ is reduced in intensity by a factor of $1/e$ through absorption or scattering, according to:
\begin{equation}
I(x) = I_0 \,\textup{exp}(-x/\lambda_{abs}) 
\end{equation}
where $x$ is the distance travelled by the photons. Sea water shows maximum transparency for $\sim 400$\,nm photons, for which one obtains an absorption length value of $\lambda_{abs}\approx 60$\,m.

For what concerns scattering, both $\lambda_s$ and the angular distribution of the momentum of the outgoing particles must be taken into account, because both contribute to the definition of an effective scattering length. The scattering typically occurs on water molecules (``Rayleigh scattering'') and on larger particulates (``Mie scattering''), resulting in a small total scattering angle over the detectable wavelength range. The effective scattering length is defined as:
\begin{equation}
\lambda^{eff}_s = \frac{\lambda_s}{1-\langle \cos\theta_s\rangle}
\end{equation}
where $\langle \cos\theta_s\rangle$ is the average scattering angle. In sea water typical values are $\lambda_s \approx 55$\,m and $\lambda^{eff}_s \approx 265$\,m, measured for photon wavelength of 470\,nm.






%\begin{thebibliography}{0}

%\bibitem{ref:CRs_chemical_composition}
%Donnelly T.\,W.\,, Formaggio J.\,A.\,, Holstein B.\,R.\,, Milner R.\,G.\,, Surrow B.\,, \emph{Chemical Composition of Galactic Cosmic Rays with Space Experiments}, New York NY

\bibliography{master_thesis}
\bibliographystyle{siam}

%\end{thebibliography}
\end{document}